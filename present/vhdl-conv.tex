
\begin{frame}[fragile]{VHDL Generierung}
Vorgehen:
\begin{enumerate}
	\item C source code compilieren zu ir format
	\item Compiliere ir format zu VHDL
\end{enumerate}

Beispiel code:
\begin{lstlisting}
variable node199 : signed(31 downto 0):= (others => '0'); 
variable node194 : signed(31 downto 0):= (others => '0'); 
variable node193 : signed(31 downto 0):= (others => '0'); 
[...]
node199 := node172 xor node198;
node194 := node199 + node193;
node193 := unsigned (resize(node194,32));
\end{lstlisting}
\end{frame}

\begin{frame}[fragile]{VHDL Generierung}
Idee:
\begin{enumerate}
	\item Nutzung der minimalen variablen breite
	\item Dadurch verminderte LUT anzahl
\end{enumerate}

Beispiel code:
\begin{lstlisting}
variable node199 : signed(9 downto 0):= (others => '0'); 
variable node194 : signed(16 downto 0):= (others => '0'); 
variable node193 : signed(31 downto 0):= (others => '0'); 
[...]
node199 := resize(node172,10) xor resize(node19,10);	
node194 := resize(node199,17) + resize(node193,17);	
node193 := resize(unsigned(resize(node194,32)),32);	
\end{lstlisting}
\end{frame}

\begin{frame}{VHDL Generierung}

\textbf{Vorgehen der Evaluation:}
\begin{enumerate}
	\item Übersetzen eines C source code zu VHDL ohne Optimierung
	\item Nochmaliges übersetzen mit Optimierung
	\item Vergleich der beiden Ergebnisse
\end{enumerate}
\end{frame}

\begin{frame}{VHDL Generierung}
\begin{tabular}{c | c | c | c}
	IDE    & Unoptimiert & Optimiert & Handoptimiert \\
	\hline
	Vivado & 108 & 112 & 109 \\
	Quartus & 54 & 54 & - \\
\end{tabular}
\textbf{Statistik des Testcodes:} \newline
\begin{tabular}{ c c c }
	& Unoptimiert & Optimiert \\
	Genutzte Bitbreite & 928 & 609 \\
\end{tabular}
\end{frame}
