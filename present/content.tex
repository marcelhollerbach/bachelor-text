
\begin{frame}[fragile]{Das Ziel}
\begin{lstlisting}
int arr[4];

for (int i = 0; i < 4; i++) {

	int x = i*i;      
	int y = (i << 4); 
	int res = x + y;  
	arr[i] = res;
}
\end{lstlisting}
Wie viele Bits nutzen \textit{i}, \textit{x}, \textit{y}, \textit{res}? \newline
\end{frame}

\begin{frame}{Was ist eine Bitbreite?}
\begin{center}
	\begin{definition}{Bitbreite}
		(stable\_bits, is\_positive)
	\end{definition}
\end{center}

\begin{tabular}{ c c | c c c c c c c c | c c }
	mode & value & \multicolumn{8}{c|}{bit representation} & stable\_bits & is\_positive \\
	\hline
	\hline
	Bs & 5     & \color{red}0 & \color {red}0 & \color{red}0 & \color{red}0 & 0 & 1 & 0 & 1 & 4 & true\\
	Bs & -2    & \color{red}1 & \color{red}1 & \color{red}1 & \color{red}1 & \color{red}1 & \color{red}1 & 1 & 0 & 6 & false\\
    Bu & 5     & \color{red}0 & \color{red}0 & \color{red}0 & \color{red}0 & \color{red}0 & 1 & 0 & 1 & 5 & - \\
\end{tabular}

\end{frame}

%\begin{frame}[fragile]{Das Ziel}
%\begin{lstlisting}
%\begin{tabular}{ c c }
%Variable & Verbrauch \\
%i & 3 \\
%x & 6 \\
%y & 7 \\
%res & 8 \\
%\end{tabular}
%\end{lstlisting}
%\end{frame}

\begin{frame}{libFIRM}
\begin{columns}
	\begin{column}{0.5\textwidth}
		\begin{enumerate}
			\item Integer-modes sind genutzt
			\item Nutzung von arithmetische/logische Ausdrücke
			\item Behandlung von Confirm-nodes
		\end{enumerate}
	\end{column}
		\begin{column}{0.5\textwidth}  %%<--- here
			\begin{center}
				\begin{tikzpicture}
				\node[graph]{
					\begin{tikzpicture}[remember picture]
					\node[block] (startblock) {
						\begin{tikzpicture}
						\node[firm]    (const0) at (0,2) {Const : 0};
						\node[phi]     (phi)    at (-1,1) {Phi};
						\node[firm]    (const1) at (1,1) {Const : 1};
						\node[firm]    (add)    at (0,0) {Add};
						\node[firm]    (cmp)        at ( 0,-1) {Cmp less\_equal};
						\node[control] (cond)       at ( 0,-2) {Cond};
						\node[control] (false)      at (-1,-3) {False};
						\node[control] (true)       at ( 1,-3) {True};
						
						\draw[dataDependency]    (add.60)      -- ++(0,0.1) -| (const1);
						\draw[dataDependency]    (add.120)     -- ++(0,0.1) -| (phi);
						\draw[dataDependency]    (phi.60)      -- ++(0,0.1) -| (const0);
						\draw[dataDependency]    (phi.120)     -- ++(0,0.1) -- ++(-1,0) -- ++(0,-2) -| (add.240);
						\draw[dataDependency]    (cmp.62)     --              (add.300);
						\draw[dataDependency]    (cond)        --              (cmp);
						\draw[controlDependency] (false.north) -- ++(0,0.1) -| (cond.240);
						\draw[controlDependency] (true.north)  -- ++(0,0.1) -| (cond.300);
						
						\end{tikzpicture}
					};
				\end{tikzpicture}
			};
		\end{tikzpicture}
		\end{center}
	\end{column}
\end{columns}
\end{frame}

\begin{frame}{libFIRM}
\begin{center}
	\begin{tabular}{c | c}
		\textbf{Confirm-node} &
		\textbf{Conv-node} 
		\\
		\begin{tikzpicture}
		\node[firm]    (add)    at (-1,1) {Add};
		\node[firm]    (const0) at (1,1) {Const : 8};
		\node[firm]    (confirm) at (0,0) {Confirm : <};
		\draw[dataDependency]    (confirm.60)      -- ++(0,0.1) -| (const0);
		\draw[dataDependency]    (confirm.120)     -- ++(0,0.1) -| (add);
		\draw[dataDependency]    (0,-1) -- (confirm);
		\end{tikzpicture}
		&
		\begin{tikzpicture}
		\node[firm]    (add)    at (0,1) {Add : \textit{Bs}};
		\node[firm]    (conv) at (0,0) {Conv : \textit{Is}};
		\draw[dataDependency]    (conv) -- (add);
		\draw[dataDependency]    (0,-1) -- (conv);
		\end{tikzpicture}
		\\
	\end{tabular}	
\end{center}

\end{frame}

