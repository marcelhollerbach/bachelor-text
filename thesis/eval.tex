\chapter{Evaluation}\label{sec:eval}

The most interesting projects for evalulating this are projects which are doing decoding of a binary blob into something like a image. For this purpose, libjpeg, libpng and libgif are evaluated. Additionally zlib is used as it is a general purpose library for lossless compression.

For those projects we are measuring two things. First we measure how effective our bitwidth saving is, and how much we have saved. In the second case we are trying to compare the assembler with and without the applied optimizations.

\section{General bitwidth}
%How much bitwidth the nodes are having in average per function
A Node itself has a bitwidth usage, if we dont perform our analysis, then we have to assume the code uses the complete bitwidth of a node. Which is the whole data word of a mode.
The bitwidth of a function then can be described by the average usage of all the nodes. Reason for using the average is, we can compare large functions with smaller ones. However, a small function like (ref) has not as much potential for analyzing its bitwidth, than some bigger function.
\paragraph{Const nodes}
\begin{figure}
	\centering
	\pgfplotstabletypeset[col sep=comma,
	columns={Library, Nodes, Ratio of Const-Nodes},
	columns/Library/.style={column type=l,string type},
%	columns/"nodes"/.style={column type=l},
%	columns/"percentNodes"/.style={column type=l},
	every head row/.style={before row=\toprule, after row=\midrule}, 
	every last row/.style={after row=\bottomrule}
	]{data.csv}
\caption{Node statistics of the project.}
\label{table:amount_const}
\end{figure}
While evaluating the projects, it came up that a normal graph consists of a lot of constant nodes, exact number can be found in \autoref{table:amount_const}. The problem with this is, the bitwidth of such a node can be quite low, while it does not matter on the hardware at all, since they are often decoded directly in a hardware instruction, or don't take up any additional space in vhdl. Additionally, calculating how much place a constant takes, is a matter of one log2(...) call. And can safe up a lot of bitwidth when the number is low. However, we are more interested in what the fixed point iteration safes. Due to the iterations, not based on constant values.

%Excluding const nodes
%results without const
%results with const

\section{Optimization assembler improvements}
\section{Optimization vhdl improvements}
%