\chapter{Evaluation}\label{sec:eval}

Hier wird nun das Ergebnis des vorherigen Kapitels kritisch betrachtet.
Verbesserungen werden anhand von konkreten Experimenten und Zahlen belegt.
Eine saubere statistische Auswertung ist das Ziel.

Für schöne Tabellen ist das \enquote{booktabs} package zu empfehlen.
Ein Beispiel ist in \cref{fig:example_table} zu sehen.

\begin{figure}[hb]
\begin{center}
\begin{tabular}{lrrrr}
\toprule
Fach & xkcd Comics & Spaß & $\sigma$ & p \\
\midrule
Informatik & 1325 &  100\% & 12,3 & 3\% \\
Physik     & 1324,31 &  87\%& 1,733 & 0.03\%  \\
Geologie & 123 & 23\% & 1,3 & 11\% \\
Wirtschaft & 5 & 4\% & 12 & 1\% \\
Deutsch & 0 & 101 & 92,3 & 33\% \\
\midrule
Geom. Mittel & 743 & 63\% & 12,3 & 3\% \\
\bottomrule
\end{tabular}
\end{center}
\caption{Eine Beispieltabelle.
Man beachte, dass zwischen Datenzeilen keine Linien sind.
Außerdem ist der Beschreibungstext hier sehr ausführlich,
damit der Leser nicht den zugehörigen Abschnitt im Fließtext finden muss.
}
\label{fig:example_table}
\end{figure}

Zum Benchmarken empfehlen wir das Tool \enquote{temci}~\cite{temci} und ein Studium der zugehörigen Bachelorarbeit~\cite{bechberger16bachelorarbeit}.

\section{General runtime}
\section{Improvements over VRP}
\section{Improvements for vhdl generation}