\chapter{Introduction}\label{sec:intro}

The problem to solve is called bitwidth analysis.
The bitwidth with a data word can be seen as the bits that are actually used in the runtime of the source code. Lets take the following example:
\begin{lstlisting}[frame=single]
int arr[4];

for (int i = 0; i < 4; i++) {
  int res = (i << 4) + i*i; 
  arr[i] = res;
}
\end{lstlisting}

After running the bitwidth analysis, every operation has a attached information structure which indicates how many bits are actually used by the code. The developed algorithm applied to the the code results in something like this:
\linebreak
$ i \in {0..3}$;
$ res \in {0..90}$; 
