\chapter{Einführung}\label{sec:intro}

%In diesem Kapitel wird das Problem vorgestellt, das diese Arbeit löst.
%Es sollte verständlich sein (anschauliches Beispiel?).
%Das Problem sollte wichtig sein,
%denn das motiviert weiterzulesen.
%Bestenfalls ist dieses Kapitel auch für Laien verständlich.

The problem to solve is called bitwidth analysis.
The bitwidth with a dataword can be seen as the bits that are actually used in the runtime of the sourcecode.
Let me give you an example:
\begin{lstlisting}[frame=single]
int arr[4];

for (int i = 0; i < 4; i++) {
   arr[i] = pow(2, i);
}
\end{lstlisting}
the variable i is actually only used for counting from 0 to 3.
Thus we only actually use two bits. The other bits are unused, and remain 0.


%FIXME for loop code with a int
