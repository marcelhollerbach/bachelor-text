\chapter{Appendix}

\section{Table of node rules}

\label{table_of_rules}

A operation has operands, all of them are referred to as \textit{operands}, the first is called \textit{a}, the second is called \textit{b}. In the first column the formula for the new stable digits is displayed. The second column is the formula for the is\_positive flag. $n_{stable}$ is used in the second column to refer to the value calculated in the first column.

\begin{tabular}{ l | c | c }
	op & used bits & is\_positive \\
	\hline
	add & $max(min(a_{stable}, b_{stable}) - 1, 0)$ & $a_{positive} \wedge b_{positive} \wedge n_{stable} > 0 $\\
	minus &  $max(a_{stable} - 1, 0) $ & $\textit{false}$\\
	sub &  $max(min(a_{stable}, b_{stable}) - 1, 0) $ & $\textit{false}$\\
	mul &  $max(b - mode + a_{stable}, 0)  $ & $a_{positive} \wedge b_{positive} \wedge n_{stable} > 0 $\\
	div &  
	$ 
	\left\{
	\begin{array}{l}
	a_{stable}\\ 
    max(a_{stable} - 1, 0)
	\end{array}
	\begin{array}{l}
	, mode\text{ is signed} \\ 
	, \text{otherwise}.
	\end{array}
	\right.$
	& $a_{positive} \wedge b_{positive} \wedge n_{stable} > 0 $ \\
	mod &  & \\
	\hline
	shl & $a_{stable} - b_{stable}  $& 
	$ 
	\left\{
	\begin{array}{l}
	a_{positive}\\ 
	\text {false}
	\end{array}
	\begin{array}{l}
	, n_{stable} > 0 \\ 
	, \text{otherwise}.
	\end{array}
	\right.$
	 \\
	shr & $a_{stable} + b_{stable}  $& $\textit{true}$\\
	shrs & $a_{stable} + b_{stable}  $& $a_{positive}$\\
	\hline
	conv &  $ max(a_{stable} + (mode - old\_mode), 0)  $ & $a_{positive} \wedge n_{stable} > 0 $\\
	\hline
	$
	\begin{array} {l}
        \text{max} \\
        \text{phi} \\
        \text{and} \\
        \text{eor} \\
        \text{or} \\
	\end{array}$ & $min(operands)$ & $\bigwedge(operands_{positive})$ \\
	\hline
\end{tabular}
\section{Evaluation C source code}
\begin{figure}
	\centering
	\begin{lstlisting}[frame=single]
	uint32_t test_atom(uint32_t input0, uint32_t input1) {
	int16_t inp0, inp1, inp2, inp3,
	abs0, abs1, abs2, abs3;
	#define ABS(x) ( x<0 ? -x : x )
	
	inp0 = input0;
	inp1 = input0 >> 16;
	inp2 = input1;
	inp3 = input1 >> 16;
	
	abs0 = ABS(inp0);
	abs1 = ABS(inp1);
	abs2 = ABS(inp2);
	abs3 = ABS(inp3);
	return abs0 + abs1 + abs2 + abs3;
	}
	
	\end{lstlisting}
	\caption{Test code for measuring VHDL generation improvements}
	\label{code:vhdl-eval}
\end{figure}