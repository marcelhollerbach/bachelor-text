\begin{figure}
	\centering
	\begin{tikzpicture}[scale=1.0, transform shape]
		\node[graph]{
			\begin{tikzpicture}[remember picture]
			\node[block] (startblock) {
				\begin{tikzpicture}
				\node[firm]    (cmp)        at ( 0,2) {Cmp};
				\node[control] (cond)       at ( 0,1) {Cond};
				\node[control] (false)      at (-1,0) {Proj};
				\node[control] (true)       at ( 1,0) {Proj};
				
				\draw[dataDependency]    (cmp.120)     -- ++(0,0.1) -| (-1, 3);
				\draw[dataDependency]    (cmp.60)      -- ++(0,0.1) -| ( 1, 3);
				\draw[dataDependency]    (cond)        --              (cmp);
				\draw[controlDependency] (false.north) -- ++(0,0.1) -| (cond.240);
				\draw[controlDependency] (true.north)  -- ++(0,0.1) -| (cond.300);
				\end{tikzpicture}
			};
			\node[block, anchor=north east] (left) at ($(startblock.south) + (-1,-0.5)$) {
				\begin{tikzpicture}
				\node[const] (c0) at (0,0) {true-block};
				\end{tikzpicture}
			};
			\node[block, anchor=north west] (right) at ($(startblock.south) + (1,-0.5)$) {
				\begin{tikzpicture}
				\node[const] (c1) at (0,0) {false-block};
				\end{tikzpicture}
			};		
			% Control dependencies
			\begin{scope}[every path/.style = {controlDependency}]
				\draw (left.north)  -- ++(0,0.2) -| (false);
				\draw (right.north) -- ++(0,0.2) -| (true);
			\end{scope}
	
			\end{tikzpicture}
		};
	\end{tikzpicture}
\caption{The construct that is called a upper bound compare node}
\label{fig:compare}
\end{figure}
