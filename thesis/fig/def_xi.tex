\begin{figure}
	\centering
	\begin{tikzpicture}
		\node[graph]{
			\begin{tikzpicture}[remember picture]
				\node[block] (startblock) {
					\begin{tikzpicture}


						\node[firm]    (const1) at (3,3) {Const};
						\node[firm]    (const2) at (3,2) {Const};
						\node[firm]    (mul)    at (1,2) {Mul};
						\node[firm]    (add)    at (1,1) {Add};
						\node[firm]    (sub)    at (0,0) {Sub := $\lambda$};
						\node[firm]    (mul2)    at (-0.5,3) {Mul};
						\node[firm]    (mod)    at (1,3) {Mod};

						\draw[dataDependency]    (sub.60)      -- ++(0,0.1) -| (add);
						\draw[dataDependency]    (sub.north)   -- ++(0,0.1) -| (mul2);	
						\draw[dataDependency]    (add.north)     -- ++(0,0.1) -| (mul);
						\draw[dataDependency]    (add.60)      -- ++(0,0.1) -| (const2);
						\draw[dataDependency]    (mul.60)     -- ++(0,0.1) -| (const1);
						\draw[dataDependency]    (mul.north)   -- ++(0,0.1) -| (mod);
						
						\draw[red, thick,dotted]     ($(mul.north west)+(-0.15,0.15)$) rectangle ($(add.south east)+(0.15,-0.15)$);
					\end{tikzpicture}
				};
			\end{tikzpicture}
		};
	\end{tikzpicture}
\caption{A subgraph of a FIRM graph. The result of $\xi(\lambda)$ is highlighted in red}
\label{fig:def:xi}
\end{figure}