\begin{figure}
	\centering
	\begin{tikzpicture}
		\node[graph]{
			\begin{tikzpicture}[remember picture]
				\node[block] (startblock) {
					\begin{tikzpicture}
					\node[firm]    (add1)    at (-2,2) {Add};
					\node[firm]    (add2)    at (0,2) {Add};
					\node[firm]    (mul)    at (-1,1) {Mul};
					\node[firm]    (const3) at (1,1) {Const};
					\node[firm]    (sub)    at (0,0) {Sub};
					
					\draw[dataDependency]    (mul.60)      -- ++(0,0.1) -| (add2);
					\draw[dataDependency]    (mul.120)      -- ++(0,0.1) -| (add1);
					\draw[dataDependency]    (sub.60)      -- ++(0,0.1) -| (const3);
					\draw[dataDependency]    (sub.120)      -- ++(0,0.1) -| (mul);
					
					\end{tikzpicture}
				};
			\end{tikzpicture}
		};
	\end{tikzpicture}
	\caption{Possible loop configurations}
\label{fig:arithmetical_chains}
\end{figure}